\documentclass[11pt]{article}

\usepackage{url}

%%  Dimensions
%%
\topmargin             0in
\headheight            0in
\headsep               0in
\textheight          8.5in
\textwidth           6.5in
\evensidemargin        0in
\oddsidemargin         0in

%%  Definitions
%%
\renewcommand{\baselinestretch}{1.1}
\pagestyle{plain}


\begin{document}

\begin{center}
{\bfseries \LARGE ECEN 489: Task 3 -- Microcontroller Software\\[8mm]}
\end{center}


\section{Arduino}

Arduino is an open-source electronics prototyping platform designed for simplicity.
An interesting aspect of the Arduino platform is the standard way in which the connectors are exposed, allowing the CPU board to be connected to a variety of interchangeable add-on modules known as shields.
The software consists of a standard programming language compiler and the boot loader that runs on the board.
The Arduino is programmed using a language similar to C++, with some simplifications.
This is accomplished using a Processing-based integrated development environment.
The primary website for Arduino is listed below.
\begin{center}
\url{http://www.arduino.cc/}
\end{center}
This site hosts step-by-step instructions for setting up the Arduino software and connecting it to a prototyping platform.
It also offers a comprehensive set of tutorials and examples.
Finally, the site provides an online reference for the Arduino language.

\section*{Action Items}

\begin{itemize}
\item \textbf{Check Out:} Arduino microcontroller.
\item \textbf{Read:} Getting Started with Arduino. \\
\url{http://arduino.cc/en/Guide/HomePage}
\item \textbf{Download and Install:} The Arduino software (latest release). \\
\url{http://arduino.cc/en/Main/Software}
\item \textbf{Read:} BareMinimum. \\
\url{http://arduino.cc/en/Tutorial/BareMinimum}
\end{itemize}


\end{document}

