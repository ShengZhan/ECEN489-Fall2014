\documentclass[11pt]{article}

\usepackage{url}

%%  Dimensions
%%
\topmargin             0in
\headheight            0in
\headsep               0in
\textheight          8.5in
\textwidth           6.5in
\evensidemargin        0in
\oddsidemargin         0in

%%  Definitions
%%
\renewcommand{\baselinestretch}{1.1}
\pagestyle{plain}

\begin{document}

\begin{center}
{\bfseries \LARGE ECEN 489: Task 1 -- Necessary Software\\[8mm]}
\end{center}


\section{The C++ Programming Language}

C++ is a popular, general-purpose programming language.
It is often considered an intermediate-level language because it admits both high-level and low-level paradigms.
C++ adds object-oriented programming features to C.
It supports classes and their main attributes: abstraction, encapsulation, inheritance, and polymorphism.
Technically, C++ incorporates the C standard library, with slight modifications.
Pertinent documentation about this language can be found on the C++ resources network.
\begin{itemize}
\item \url{http://www.cplusplus.com/}
\end{itemize}
Many compilers are available for C++, including Visual Studio (Microsoft) and GCC (GNU Project).
The C/C++ syntax forms the basis for software development on several microcontroller platforms.
Texas A\&M University provides access to several books on C++ programming.
For example, \emph{The C++ programming language} is accessible online through campus.
\begin{itemize}
\item \url{http://proquest.safaribooksonline.com/9780133522884}
\end{itemize}
To build and debug a C++ projects, a toolchain is required.
Each platform that runs a C/C++ Development Tools (CDT) requires a unique installation process.


\subsection*{Action Items}

\begin{itemize}
\item \textbf{Download and Install:} A C++ Toolchain.
\begin{itemize}
\item \textbf{Windows Option 1:} Microsoft \texttt{Visual Studio}, which is available for free to ECE students through Microsoft DreamSpark.\\
\url{https://www.dreamspark.com/Institution/Access.aspx}
\item \textbf{Windows Option 2:} MinGW using the \texttt{mingw-get-inst} package and default directory.\\
\url{http://sourceforge.net/projects/mingw/files/}
\item \textbf{MacOS X:} Apple GNU toolchain included in the Xcode IDE.\\
\url{http://developer.apple.com}
\item \textbf{GNU/Linux:} Most GNU/Linux distributions provide the GNU toolchain.\\
\url{https://gcc.gnu.org}
\end{itemize}
\item \textbf{Create, Build, and Run:} HelloWorld.
\end{itemize}


\newpage
\section{CMake}

CMake is a cross-platform, open-source build system.
It is a family of tools designed to build, test and package software.
CMake is used to control the software compilation process using simple platform and compiler independent configuration files.
CMake generates native makefiles and workspaces that can be used in various compiler environments, and supports cross-platform development.
\begin{itemize}
\item \url{http://www.cmake.org/}
\end{itemize}


\subsection*{Action Items}

\begin{itemize}
\item \textbf{Read:} The CMake tutorial. \\
\url{http://cmake.org/cmake/help/cmake_tutorial.html}
\item \textbf{Peruse:} The Introduction to CMake Course. \\
\url{http://cmake.org/cmake/resources/webinars.html}
\item \textbf{Download and Install:} CMake. \\
\url{http://cmake.org/cmake/resources/software.html}
\item \textbf{Create, Build, and Run:} HelloWorld using CMake.
\end{itemize}


\subsection*{Structure}

Version control systems form a great paradigm to facilitate collaborative projects.
Still, one of the many difficulties encountered in large projects is the fact that most programmers have a strong preference for a programming style.
Following a few elementary guidelines can greatly simplify team work.
A short list of generally applicable rules can be found below.
\begin{itemize}
\item \textbf{Main:} Only create one \texttt{main()} class to avoid confusing automated builders.
This will prevent the \texttt{duplicate symbol} error that may otherwise occur at \texttt{link} time.
\item \textbf{Structure:} Declare your classes in \texttt{header} files and define your classes in \texttt{source} files in a structured manner.
\item \textbf{Hierarchy:} For simple projects, place all your files in a same directory, e.g., \texttt{src}.
\item \textbf{Documentation:} Annotate your code appropriately, and remember that comments should be decipherable by someone else than the original programmer.
\item \textbf{Libraries:} Only include cross-platform libraries for greater compatibility, e.g., \texttt{Boost}.
\end{itemize}


\newpage
\section{Integrated Development Environments (optional)}

An integrated development environment (IDE) is a software application that provides comprehensive tools for software development.
An IDE often provides a source code editor, build automation tools, and a debugger.
A major benefit of IDEs is a significant reduction in establishing proper build configurations.


\subsection*{The Eclipse Foundation}

Eclipse is a community for individuals and organizations who wish to collaborate on commercially-friendly open source software.
The Eclipse IDE can be used to develop applications in Java, C++, Python and other languages.
Eclipse is a powerful platform, and it can be used to program C++.
The standard C/C++ Development Tools (CDT) in Eclipse supports integration with the GNU toolchain, which includes the \texttt{make}, \texttt{gcc}, and \texttt{gdb} utilities.

\subsection*{Alternate Programming Environments}

There are many alternate IDEs for C++ development.
This includes Microsoft Visual Studio, Apple Xcode, NetBeans.
In addition, code can be written in code editors such as Notepad++, TextWrangler, GNU Emacs, and Vim.

\subsection*{Action Items}

\begin{itemize}
\item \textbf{Read:} About the Eclipse Foundation. \\
\url{http://www.eclipse.org/org/}
\item \textbf{Read:} C/C++ Development User Guide. \\
\url{http://help.eclipse.org/luna/index.jsp}
\item \textbf{Download and Install:} A Java Runtime Environment (JRE). \\
%This is needed to run Eclipse IDE. \\
\url{http://www.oracle.com/technetwork/java/javase/downloads/index.html}
\item Option~1 -- \textbf{Download and Install:} Eclipse IDE for C/C++ Developers (latest release). \\
\url{http://www.eclipse.org/downloads/}
\item Option~2 -- \textbf{Download and Install:} Eclipse Standard (latest release). \\
Under \texttt{Install New Software > Programming Languages}, select the \texttt{C/C++ Development Tools} and \texttt{C/C++ Development Tools SDK} packages.
\end{itemize}


\end{document}

