\documentclass[11pt]{article}

\usepackage{times}

%%  Dimensions
%%
\topmargin             0in
\headheight            0in
\headsep               0in
\textheight          8.5in
\textwidth           6.5in
\evensidemargin        0in
\oddsidemargin         0in
\parindent             0in

%%  Definitions
%%
\renewcommand{\baselinestretch}{1.1}
\pagestyle{plain}


\begin{document}

\begin{center}
{\bfseries \LARGE Opt-Out Challenge 1}
\end{center}

\noindent
\rule[1mm]{\linewidth}{0.2pt}


Consider a text file that contains two top-level fields: \texttt{location} and \texttt{weather}.
The first field is to remain untouched.
The second field, \texttt{weather}, contains two attributes, \texttt{temp} and \texttt{tempunit}.
The latter attribute can be Kelvin, Celsius or Fahrenheit.
Your C++ program should be able to read the file; convert the second field to a temperature in Kelvin (both \texttt{temp} and \texttt{tempunit} should be changed if needed); and modify the original text file accordingly.
For convenience, a sample file appears below.

\begin{verbatim}
{
"location":
   {
      "country":"United States",
      "state":"Texas",
      "city":"College Station"
   },
"weather":
   {
      "temp":97,
      "tempunit":"Kelvin"
   }
}

\end{verbatim}

\end{document}
